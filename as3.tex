\documentclass{article}
\usepackage{graphics,graphicx}
\usepackage{hyperref}
\usepackage{fullpage}
\begin{document}
\title{CS4482: Assignment 3}
\author{Kartik Thakore | kthakore@uwo.ca}
\maketitle

\section{Completed Features}
\begin{description}
\item [Dynamic Scene graph]
The scene graph is used only to update object locations and rotations based on the parent. This allows distance based frustum culling as mentioned below.
All nodes have their own spatial locations, and relative locations. This allows a user to change a node based on local coordinates. 
\item [Frustum Culling]
The frustum culling is done using a bounding sphere. Currently I use the inverse of the model view matrix to get the transformation matrix, which is then 
applied to the bounding sphere location. However this can be done by having a software matrix stack, which I began to implement in matrix.c, and can be seen 
in \verb object.c \verb obj_displace_bb() . 
\item [Locality of Information]
Additionally all objects are store in an indexed linear array to give O(1) retrieval performance gain. However this sacrifices memory a bit, but that is fine for 
a desktop PC, which was the aim of this scene graph. Additionally after the scene graph traversals are done, objects are 'registered' into a linked list that is
ordered by distance. This allows me to render object closest to the camera up to the polygon limit. 
\item [Usage]
Drag the mouse with left click button, press WASD.
\item{description}

\end{enumerate}

\section{Game Files Locality on DVD/CD devices} 

\begin{description}
\item[Information Required:]
First I would require the information on the seeking algorithm of a DVD device. I need to know what points of data take the least time to traverse between, and what 
takes more time. Next I would need to analyze the times files are accessed in a typical game play. Also when the files are loaded in terms of game loop setup and running
is very important to know. The algorithm should also take into consideration the size of the file to be loaded, and will the DISC medium fragment it to store it across
tracks.  

\item[Acquiring the Information:]
The DVD/CD information will probably be acquired from the manufacturer of the DVD rom, assuming we have control over where the game disc is used. In terms of how
often and when a game file is loaded, static analysis of the program will need to be done. 
\item[Calculations Required:]
First the bare acceptable speed for file loading in the game loop. This should be our aim, however it doesn't necessarily have to be by time, but rather CPU operations.
Additionally we will need the comparable CPU clock rates require to load our files. At which time we can decided to stuff like worker threads or straight loading.
\item[Algorithm Overview:]
So with this information here is the following Algorithm:
\end{description}

\begin{verbatim}

FOR GAME CALLS TO FILE:
	IF FILE == GAME_LOOP_FILE
	    FILE.PRIORITY++ 
	ELSE
	    FILE.PRIORITY--
	IF FILE.SIZE <= ONE_TRACK.SIZE
	    FILE.PRIORITY--
	ELSE
	    FILE.PRIORITY++

SORT FILES BY FILE.PRIORITY

SORT TRACKS BY TRACK.READ_SPEED

REVERSE TRACKS

TRACKS[0].QUEUE = FILES

FOREACH TRACK in DISC
    FOREACH FILE in TRACK.QUEUE
        TRACK.ADD( QUEUE.FILE )
        IF TRACK.FULL
            NEXT_TRACK.QUEUE.ADD = TRACK.QUEUE
	IF FILE.SIZE >= TRACK
            NEXT_TRACK.QUEUE.ADD = FILE
	ELSE
            TRACK.ADD(FILE)

\end{verbatim}


\end{document}
